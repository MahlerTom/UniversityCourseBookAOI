\chapter{Inserting Images} \label{c2_secondchapter:cha}

\section{Images}\label{c1_images:sec}

As in Word, in \LaTeX{}, images are separate from the text. Images are usually
packaged together with a caption and a label to reference it from the text.
These three entities are packaged together into a figure. The figure itself
configures the size of the image as well as where it should be put. Let us look
at a code sample:
\begin{lstlisting}[language=Tex]
\begin{figure}[H]
    \centering
    \includegraphics{images/ebookLatex_Cover.jpg}
    \caption{The cover of this book.} \label{c1_cover:fig}
\end{figure}
\end{lstlisting}

Let us go through this line by line. At the core is the image, included with
\lstinline[language=Tex]!\includegraphics{path to file}!. It inserts the image
specified by the ``path to file.'' With the
\lstinline[language=Tex]!\adjustbox{}! command, we can adjust the image size
according to the page width (\lstinline[language=Tex]!\columnwidth!) and page
height (\lstinline[language=Tex]!\textheight!). 

Below there is the caption and the label. \LaTeX{} automatically numbers each
figure, so in the text, we can later refer to it with
\lstinline[language=Tex]!\ref{c1_cover:fig}! which prints out the number of the
figure. Finally, all these commands are centered with the
\lstinline[language=Tex]!\centering! command and surrounded with the figure
environment. The \lstinline[language=Tex]![!ht]! instructs \LaTeX{} to try to
place the image exactly where it is in the \LaTeX{} code.

\begin{figure}[!ht]
	\centering
	\includegraphics{images/cover.jpg}
	\caption{The cover of this book.} \label{c1_cover:fig}
\end{figure}

In Figure~\ref{c1_cover:fig}, you can see the result of the command. Instead of
graphics, you can also include other TEX files that contain graphics (or
commands to draw graphics, see chapter~\ref{c1_tikzgraphics:sec}).


\section{TikZ Graphics}\label{c1_tikzgraphics:sec}

For graphics, you can use the inbuilt TikZ graphics generator. Due to its
flexibility, I even recommend images you already have for a number of reasons:

\begin{itemize}
    \item TikZ graphics can very easily changed (especially for for example
    translations or making corrections).
    \item TikZ graphics are small and flexible. They can be easily scaled to any
    size and are directly integrated into your project (no time-consuming
    editing in an external graphics program necessary).
    \item TikZ graphics look better. As vector graphics are sent directly to the
    printer, we need not to worry about readability.
\end{itemize}

If you want to create a TikZ graphic, simply create a new TEX file in the
\textit{tex-images} folder and include it with \lstinline[language=Tex]!\input!
(replacing \lstinline[language=Tex]!\includegraphics{}!) where you want to. 

Then, do a ``recompile from scratch'' by clicking on the top right corner of the
preview window (showing Warning or Error) to regenerate the TikZ file. If
``up-to-date and saved'' is shown, delete the \textit{tikz-cache} directory and
recreate it. 

For the format of the file itself, it is a series of commands surrounded by the
\lstinline[language=Tex]!\begin{tikzpicture}...\end{tikzpicture}! environment.
Discussing all the commands is beyond the scope of this book, so I recommend
three options:

\begin{itemize}
    \item Check out the PGF manual at \url{https://www.ctan.org/pkg/pgf}. It is
    more than 1100 pages full with documentation of each command and
    corresponding examples.
    \item Check out the few example TikZ pictures from my two books \cite{PFH1E}
    and \cite{PFH2E} in the \textit{tex-images} directory.
\end{itemize}