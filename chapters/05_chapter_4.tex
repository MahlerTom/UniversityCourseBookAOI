\chapter{Power/EM I} \label{c4_forthchapter:cha}

\section{Electronic Circuits}
\subsection{A basic electronic circuit}
The most basic electronic circuit consist of a power supply 
(i.e. a battery) and an electrical load (any component consuming electric power) 
connected to it on one side and to the "ground" (the reference point from 
which voltages are measured) on the other side.
% here we should add a figure of this basic circuit, as shown on Yossi's presentation
The difference in the electric potential between the power supply and the ground
creates an electric current which flow through the load to the ground.

The difference in electric potential between two points is measured in Volts 
(usually denoted by \textbf{\textit{V}}). The amount of current flowing thru
the circuit at a given time is measured in Amperes (denoted by \textbf{\textit{A}}).
The electrical resistance of the load is a measure of its opposition to the flow 
of electric current through it. It is measured in Ohms (and denoted by \textbf{\textit{R}}).

% Ohm's law: V=I*R - we should add it inside a box
$V=I*R$
\newline
Ohm's law defines the relationship between the Voltage, Current and Resistance in a circuit:
The voltage is equal to the current multiplied by the resistance of the load.
Since in most of the circuits we are using, the voltage is fixed 
(defined by the characteristics of the power supply), a change in the resistance
of the circuit will cause a change in the current in the opposite direction.
This means we can measure the current over time in order to calculate the resistance.

A useful analogy for the relations between V, I and R is to imagine a fountain on a high mountain,
where the water flow down through a river to the sea. The difference in height between the fountain
and the sea is the Voltage, the width of the river can be thought of as the resistance,
and the flow of the water is the current.

$Power = Work / Time$
\newline
% Power consumption: P=I*V - we should add it inside a box
Power consumption: $P=I*V$
\newline
Electricity can be used to do various kinds of work:
\begin{itemize}
    \item Electromagnetic work (light a bulb, transmit a WiFi signal)
    \item Thermal work (heating)
    \item Mechanical work (spin a motor, vibrate a speaker)
    \item Chemical work (charging a battery)
    \item Computational work (store or load from memory, compute a value)
\end{itemize}

The power consumption of a device is the work it does divided by time.
It is measured in Watts (\textbf{\textit{W}}).
The power consumption can be calculated as current (\textbf{\textit{I}})
multiplied by Voltage (\textbf{\textit{V}}).
Power is consumed when it leaves the circuit.

\subsection{Current and Voltage dividers}
Before we take a look at two simple electronic circuits, we need to
introduce two additional terms:
A \textbf{short circuit} is a piece of wire with almost no resistance at all.
An \textbf{open circuit} is a circuit which doesn't allow any
current to pass through it.

\subsubsection{Connecting in serial}
% diagrams of connecting a short circuit and an open circuit in serial to the load.
If we connect a short circuit after the load, it will have no influence on it.
If we connect an open circuit after the load, it will increase the resistance
to a very high value, causing the current to become zero effectively.

\subsubsection{Connecting in parallel}
% diagrams of connecting a short circuit and an open circuit in parallel to the load.
If we connect an open circuit in parallel to the load, the current will flow only thru the load path,
so the current on the open circuit will be 0.
However, the voltage drop between both points of the open circuit will be the same as the drop between
the load sides.

If we connect a short circuit in parallel to the load, the current will "prefer" flowing thru it rather
than thru the load, so the current thru the load will be equal 0, while the current thru the short
circuit will be very high - by Ohm's law.
Since the cable is not a perfect conductor, some of the energy will be consumes in the form of
thermal work, so the cable will heat, and possibly melt and start a fire\dots

\section{Measuring Power Consumption}
\subsection{Ammeter}
% an icon of Ammeter symbol ("A" inside a circle) - like here: https://thenounproject.com/term/ammeter/96074/
% maybe also a good photo of such device
An Ammeter (from \textbf{Am}pere \textbf{Meter}) is a device capable of measuring the amount of
electric current going through it. It has very low resistance, so it doesn't interrupt the system
connected to it.

\subsubsection{Using Ammeter to measure power consumption}
We need to "cut" the wire connected to the load and connect both sides to the Ammeter.
Doing so will cause all current flowing through the load to pass through the Ammeter as well,
so we will be able to read the current at any given time.
In case we know the voltage (i.e. a 5V battery, a 220V power socket), we can compute the power
consumption: $P=I*V$

The problem: sometimes, we don't want (or simply can't) cut the circuit after the load in order
to connect an Ammeter.
% Yossi started talking about what to do if we don't want to cut, but the conclusion
% (after talking about connecting a Voltmeter in parallel) was we can't do it without cut...

\subsection{Voltmeter}
% an icon of Voltmeter symbol ("V" inside a circle) - like here: https://commons.wikimedia.org/wiki/File:Voltmeter_symbol.png
% maybe also a good photo of such device
A Voltmeter is a device capable of measuring the electrical potential (also called "Voltage drop")
between two points in an electric circuit.

By connecting a Voltmeter in parallel with a very small and accurate resistor, we can measure the
electric current by Ohm's law: $I=V/R$

Summary: we learnt what is Power Consumption and how we can measure it.
A very important fact is that \textbf{Power Consumption varies with time!}.
If we can find a relationship between the secret information we want to extract and
the power consumption, we can recover this information by measuring the power consumption over time.

% TO BE CONTINUED: 
% https://www.temi.com/editor/t/iT2GGbpWLF-dheqB2Wljm3PU5R9Yqyu1ynigOlYve96G2opOgaLbI27Ilw944MYylUBuF6U_xc5uZKfvlG3VhCKYswg?loadFrom=DeliveryEmail
% 2nd lecture

