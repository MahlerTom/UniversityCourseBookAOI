%%%%%%%%%%%%%%%%%%%%%%%%%%%%%%%%%%%%%
% Read the /ReadMeFirst/ReadMeFirst.tex for an introduction. Check out the accompanying book "Better Books with LaTeX" for a discussion of the template and step-by-step instructions. The template was originally created by Clemens Lode, LODE Publishing (www.lode.de), mail@lode.de, 8/17/2018. Feel free to use this template for your book project!
%%%%%%%%%%%%%%%%%%%%%%%%%%%%%%%%%%%%%


% Replace Replace with Third Chapter Name
% Replace c3_thirdchapter:cha with your chapter title label (no spaces, only lower case letters)
% Replace the text below \end{chapterpage} and insert your own text.

\begin{chapterpage}{Replace with Third Chapter Name}{c3_thirdchapter:cha}

\begin{myquotation} The perfect place for an introducing quotation.\par\vspace*{15mm}
\mbox{}\hfill \emdash{}Famous Person\index{Person, Famous}
% Add the source.
%, \citetitle{bibitem}\index{@\citetitle{bibitem}} \ifxetex\label{famousperson-bibitem-quote}\else\citep[p.~123]{bibitem}\fi
\par\end{myquotation}

\end{chapterpage}

% -------------------- replace or remove text below and paste your own text ------


\begin{figure}
\centering
\begin{tikzpicture}
\node[fill=yellow!80,ellipse] (origin) {Origin};
\node[fill=blue!30,ellipse] (destination) at (15em,0) {Destination};
\path (origin) edge[->] node[above,font=\footnotesize] {the journey} (destination);
\end{tikzpicture}
\caption{TikZ drawings will be output as SVG, which should be rendered by most modern browsers.}
\end{figure}


\begin{figure}
\centering
\begin{tikzpicture}
    % Define styles of various tikz elements.
    \tikzstyle{textbox} = [rounded corners, text width=60pt, minimum height=50pt,text centered,draw=black]
    \tikzstyle{arrow} = [thick,->,>=latex]
    \tikzstyle{block} = [rectangle,textbox]
    \tikzstyle{textarr} = [rectangle,align=center,fill=white]

    \node (latex) [block] {\LaTeX{}\\document};
    \node (overleaf) [block, left=of latex] {Overleaf};
    \node (pdf) [block, above left=of overleaf] {PDF}; % xelatex
    \node (html) [block, above right=of overleaf] {HTML}; % pdfLaTeX + tex4ht
    
    \node (print) [block, above= of pdf] {Printed\\book};
    \node (mobi) [block, above left=of html] {MOBI\\(Amazon)}; % Kindle Previewer
    \node (epub) [block, above= of html] {EPUB\\(GooglePlay)}; % Calibri
    
    \draw[out=90,in=270] [arrow] (overleaf) to node[textarr] {XeLaTeX} (pdf);
    \draw[out=90,in=270] [arrow] (overleaf) to node[textarr] {pdfLaTeX\\tex4ht} (html);
    
    \draw [arrow] (latex) -- (overleaf);
    \draw [arrow] (pdf) -- (print);
    
    \draw[out=90,in=270] [arrow] (html) to node[textarr] {Kindle\\Previewer} (mobi);
    \draw [arrow] (html) -- node[textarr] {Calibri} (epub);
\end{tikzpicture}
\caption{Build chain of \textit{Overleaf} as a TikZ picture.}
\end{figure}

\begin{figure}
\centering
\input{tex-images/latexeffortcomplexity}
\caption{Comparing complexity of \textit{Word} and \textit{LaTeX} depending on the application.}
\end{figure}

\begin{figure}
\centering
\input{tex-images/leaves-golden-cut}
\caption{Example of a drawing made in TikZ.}
\end{figure}

